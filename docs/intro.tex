%ensaios de coisas não-matematicas

Tomando \textit{O Discurso do Método}, de Descartes, fundamentamos este livro em quatro preceitos:
\begin{enumerate}{1.}
\item Nunca aceitar coisa alguma como verdadeira sem que a conhecesse evidentemente como tal;
\item Dividir cada uma das dificuldades que se examinasse em tantas parcelas quantas fosse possível e necessário para melhor resolvê-las.
\item Conduzir por ordens os pensamentos, começando pelos objetos mais simples e mais fáceis de conhecer, subindo pouco a pouco até o conhecimento dos mais compostos, supondo certa ordem mesmo entre aqueles que não se precedem naturalmente uns aos outros.
\item Fazer em tudo enumerações tão completas, e revisões tão gerais, que tivesse certeza de nada omitir.
\end{enumerate}
A partir disto, e do apoio que recebemos de colegas, amigos, professores e familiares, construímos este material para auxiliar todos aqueles que escolheram a mesma jornada que nós. \par 
Temos muito a agradecer, principalmente aos professores que nos cederam materiais ou avaliaram a construção deste livro. Estes pertencem ao conjunto $A_1$.
\[A_1={Daniela Rodrigues Ribas, Vandoir Stormowski, Neda Gonçalves}\]
Além de nossos professores, muitos amigos e colegas participaram da produção deste material. Assim, foi definido o conjunto $A_2$.
\[A_2={Zé, }\] %seems a shit... TEXTO <3