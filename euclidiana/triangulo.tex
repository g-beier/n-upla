\chapter{Triângulos}
A partir de três pontos não colineares, conseguimos definir segmentos cuja intersecção, dois a dois, são suas extremidades. A partir da união destes segmentos, definimos triângulos.
\begin{df}
Dados três pontos $A,B,C$ não colineares, denominamos a união dos segmentos $\overline{AB}, \overline{BC}, \overline{CA}$ de triângulo $ABC$.
\[\textrm{triângulo } ABC = \triangle ABC = \overline{AB} \cup \overline{BC} \cup \overline{CA}\]
\end{df}
As extremidades dos segmentos que formam o triângulo são chamadas \emph{vértices}. Os ângulos $\angle ABC$, $\angle BCA$ e $\angle CAB$ são ditos \emph{opostos} aos segmentos $\overline{AC}$, $\overline{AB}$, $\overline{CB}$ (enquanto os segmentos são chamados \emph{lados}). Usualmente utilizamos a letra do vértice oposto para a medida de um segmento, ou seja, $c$ é a medida do segmento $\overline{AB}$.

\Blindtext