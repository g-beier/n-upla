\chapter{Distâncias}

Neste tópico, abordaremos algumas relações importantes para obtenção das distâncias enter elementos do $\mathbb{R}^3$.

\section{Distância entre dois pontos}

Se considerarmos dois pontos $A(x_1, y_1, z_1)$ e $B(x_2, y_2, z_2)$ do $\mathbb{R}^3$, então a distância entre estes pontos é definida por
\begin{equation}
d(A, B)=\sqrt{(x_2-x_1)^2+(y_2-y_1)^2+(z_2-z_1)^2}
\end{equation}

\begin{exemplo} Qual a distância entre os pontos $P(-1, 2, 0)$ e $Q(3, 2, -1)$?

Basta utilizar a relação acima
\begin{eqnarray*}
d(P, Q)   & = & \sqrt{(x_2-x_1)^2+(y_2-y_1)^2+(z_2-z_1)^2} \\
          & = &  \sqrt{(3+1)^2+(2-2)^2+(-1-0)^2} \\
          & = &  \sqrt{4^2+0^2+(-1)^2} \\
          & = &  \sqrt{16+0+1} \\
          & = & \sqrt{17}
\end{eqnarray*}

\end{exemplo}

\section{Distância de um ponto a uma reta}

Para obter a distância entre um ponto $P$ a uma reta $r$, precisamos considerar o vetor diretor $\vec{v}$ da reta e um ponto $A$ qualquer sobre a reta $r$. A distância entre a reta $r$ e o ponto $P$ será obtida através da da seguinte relação
\begin{equation} 
d(P, r)=\frac{\|\vec{v}\times \overrightarrow{AP}\|}{\|\vec{v}\|}
\label{distpr}
\end{equation}

\begin{exemplo} 
Determine a distância do ponto $P(1, 0, -2)$ a reta $r: \left\{
\begin{array}{l}
x=3+4t \\
y=-2-t \\
z=4+t \\
\end{array} \right.$

Observe que o vetor diretor da reta $r$ é $\vec{v}=(4,-1,1)$, e $\|v\|=\sqrt{4^2+(-1)^2+1^2}=\sqrt{16+1+1}=\sqrt{18}=3\sqrt{2}$.

Obtenha um ponto qualquer da reta $r$, por exemplo para $t=0$ temos o ponto $A(3,-2,4)$. E então o vetor $\overrightarrow{AP}$ será $\overrightarrow{AP}=P-A=(-2,2,-6)$.

$\vec{v} \times \overrightarrow{AP}= \left|
\begin{array}{ccc|cc}
\vec i & \vec j & \vec k & \vec i & \vec j \\
4 & -1 & 1 &4 & -1 \\
-2 & 2 & -6 &-2 & 2 \\
\end{array}
\right|= 4\vec{i} +22\vec{j} +6\vec{k}
$

E portanto temos $\|v \times \overrightarrow{AP}\|=\sqrt{4^2+22^2+6^2}=\sqrt{16+484+36}=\sqrt{536}=2\sqrt{134}$

E agora, utilizando a expressão \ref{distpr}, temos $d(P, r)=\frac{\|\vec{v}\times \overrightarrow{AP}\|}{\|\vec{v}\|}=\frac{2\sqrt{134}}{3\sqrt{2}}=\frac{4\sqrt{67}}{3}$.
\end{exemplo}

\section{Distância de um ponto a um plano}

A distância entre o ponto $P_0(x_0, y_0, z_0)$ e o plano $\pi: ax+by+cz+d=0$ é dada por: 
\begin{equation}
d(P_0, \pi)=\frac{|ax_0+by_0+cz_0+d|}{\sqrt{a^2+b^2+c^2}}
\end{equation}

\begin{exemplo} Obtenha a distância do ponto $A(1, 2, -5)$ ao plano $\pi: 12x+3y-4z+7=0$. 

Aplicando a relação acima, temos:
\begin{eqnarray*}
d(P_0, \pi) & = &\frac{|ax_0+by_0+cz_0+d|}{\sqrt{a^2+b^2+c^2}} \\
d(A, \pi) & = & \frac{|12x_0+3y_0-4z_0+7|}{\sqrt{12^2+3^2+(-4)^2}} \\
          & = &  \frac{|12\cdot 1+3\cdot 2-4\cdot (-5)+7|}{\sqrt{169}} \\
          & = & \frac{|12+6+20+7|}{13}= \frac{45}{13}
\end{eqnarray*}

\end{exemplo}

\section{Distância entre duas retas}

Consideremos o caso em que as duas retas $r$ e $s$ são reversas. Vamos considerar $P$ um ponto da reta $r$, e $\vec{v}_r$ o seu vetor diretor. Da mesma forma, $Q$ é um ponto da reta $s$ e $\vec{v}_s$ seu vetor diretor.

Desta forma, através das propriedades do produto misto, temos:
\begin{equation}
d(r, s)=\frac{|(\vec{v}_r, \vec{v}_s, \overrightarrow{PQ})|}{\| \vec{v}_r \times \vec{v}_s \|} \label{distrs}
\end{equation}

Se duas retas são paralelas, a distância entre elas é definida pela distância de um ponto de uma delas até a outra reta. Ou seja, se resume a um problema de distância entre ponto e reta.

\begin{exemplo} Obtenha a distância entre as retas $\displaystyle r: \frac{x-2}{3}=1-z \; ; \; y=2$ e $s:\left\{ \begin{array}{l}
y=4-2x\\
z=3x+1 \end{array} \right. $.

Na reta $r$ temos o vetor diretor $\vec{v}_r=(3,0,-1)$ e escolhemos um ponto qualquer, por exemplo, $P(2,2,1)$. Na reta $s$ temos o vetor diretor $\vec{v}_s=(1,-2,3)$ e escolhemos um ponto qualquer, por exemplo, $Q(2,0,7)$.

Assim, temos o vetor $\overrightarrow{PQ}=Q-P=(0,-2,6)$

$(\vec{v}_r, \vec{v}_s, \overrightarrow{PQ})= \left|
\begin{array}{ccc|cc}
3 & 0 & -1 & 3 & 0 \\
1 & -2 & 3 &1 & -2 \\
0 & -2 & 6 &0 & -2 \\
\end{array}
\right|= -16
$

$\vec{v}_r \times \vec{v}_s= \left|
\begin{array}{ccc|cc}
\vec i & \vec j & \vec k & \vec i & \vec j \\
3 & 0 & -1 & 3 & 0 \\
1 & -2 & 3 &1 & -2 \\
\end{array}
\right|= -2\vec{i} -10\vec{j} -6\vec{k}
$

Calculamos $\|\vec{v}_r \times \vec{v}_s\|=\sqrt{(-2)^2+(-10)^2+(-6)^2}=\sqrt{4+100+36}=\sqrt{140}=2\sqrt{35}$

E agora basta utilizar a relação \ref{distrs}, e obter a distância entre as retas $r$ e $s$:

$ d(r, s)=\frac{|(\vec{v}_r, \vec{v}_s, \overrightarrow{PQ})|}{\| \vec{v}_r \times \vec{v}_s \|}=\frac{|-16|}{2\sqrt{35}}=\frac{8}{\sqrt{35}}=\frac{8\sqrt{35}}{35}$

\end{exemplo}

\section{Distância entre uma reta e um plano}

Se uma reta e um plano não forem paralelos, então a distância ente eles será $0$ (zero), pois sempre terão um ponto de interseção.

No caso de uma reta paralela a um plano, basta considerar um único ponto da reta em questão, e calcular a distância deste ponto ao plano dado.

\section{Distância entre dois planos}

Se dois planos são paralelos, basta considerar o ponto de um dos planos, e calcular a distância deste ponto ao outro plano.

Se os planos não são paralelos, então a distância será $0$ (zero) pois os mesmos terão interseção.