\chapter{Exponencial}
\begin{df}
Seja $a$ um número real e $n$ um número natural. Potência de base $a$ e expoente $n$ é o número $a^n$, tal que:\[a^0 = 1\]
\[a^n = a^{n-1}\cdot a \textrm{,} \,\forall \, n\textrm{,} \,n \geqslant 1\]
\section{Consequências da definição}
\[a^1 = a^0 \cdot a = 1 \cdot a = a\]
\[a^2 = a^1 \cdot a = a \cdot a\]
\[a^3 = a^2 \cdot a = (a \cdot a) \cdot a = a \cdot a \cdot a\]
temos também, de modo geral, para $p \in \mathbb{N} \textrm{ e } p \geq 2$ que $a^p$ é o produto de $p$ fatores iguais a $a$.
\begin{exemplo}
\[3^0 = 1\]

\end{exemplo}
\end{df}
