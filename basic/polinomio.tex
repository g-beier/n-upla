\chapter{Polinômios}
\begin{df}
Polinômios em uma variável são séries de monômios em uma variável (termos na forma $a_nx^n$). Iremos trabalhar com polinômios de coeficientes reais, ou seja, $a_i \in \mathbb{R}$. Cada monômio é caracterizado por um coeficiente $a_n$, uma indeterminada\footnote{Utilizamos a notação ``indeterminada'' pois $x$ é um valor desconhecido, mas não definido. Como não é uma variável como nas funções, nem uma incógnita como nas equações. É uma indeterminada. Podemos definir o valor de $x$ sendo uma matriz, um vetor ou um número real, por exemplo.} $x$ e um expoente $n \in \mathbb{N}$. O grau de um polinômio é determinado pelo maior expoente $n$ tal que $a_n \neq 0$. A função grau de um polinômio é dada por $\partial (p(x))=n$.
\end{df}
\begin{exemplo}
\begin{align*}
P(x)&=7x^2-3x+9 &\partial(P(x))=2\\
Q(z)&=5z^5+4z^4+3z^3+2z^2+z-1 &\partial(Q(z))=5\\
M(a)&=3 &\partial(M(a))=0\\
S(t)&=t+1 &\partial(S(t))=1
\end{align*}
\end{exemplo}
Cada polinômio possui um valor numérico, que pode ser calculado atribuindo um valor a sua indeteminada.
\begin{exemplo}
\begin{align*}
P(2)&=7(2)^2-3(2)+9=&31\\
Q(0)&=5(0)^5+4(0)^4+3(0)^3+2(0)^2+(0)-1=&-1\\
M(1)&=3=&3\\
S(9)&=(9)+1=&10
\end{align*}
\end{exemplo}
As raízes de um polinômio são os valores que, quando atribuídos à indeterminada geram um valor numérico igual a 0. \par 
Para polinômios de primeiro e segundo grau, existem relações simples para obtermos as raízes do polinômio. Para polinômios de graus maiores, a obtenção é mais complexa.%texto
\begin{exemplo}
\[x^2-4=0 \Rightarrow S=\{-2, 2\} \]
\[x^5-3x^4+10x^3-30x^2+9x-27=0 \Rightarrow S=\{3\}\]
\end{exemplo}
Para polinômios de primeiro grau, podemos simplesmente isolar a incógnita e assim obter a raiz. \par 
Para polinômios de segundo grau, existe a relação de Bháskara.
\[x=\frac{-b\pm \sqrt{b^2-4ac}}{2a}\]
\begin{proof}
Tomemos a raiz de um polinômio de segundo grau, ou seja, a equação $ax^2+bx+c=0$ com $a \neq 0$. Assim, temos:
\begin{align*}
ax^2+bx+c&=0 \\
ax^2+bx&=-c \\
4a^2x^2+4abx&=-4ac \\
4a^2x^2+4abx+b^2&=b^2-4ac \\
(2ax+b)^2&=b^2-4ac \\
|2ax+b|&=\sqrt{b^2-4ac} \\
2ax+b&=\pm \sqrt{b^2-4ac} \\
2ax&=-b \pm \sqrt{b^2-4ac} \\
x&= \frac{-b \pm \sqrt{b^2-4ac}}{2a}
\end{align*}
\end{proof}
\section{Operações com Polinômios}
Podemos realizar operações com polinômios de forma similar as operações utilizadas entre seus coeficientes. Para isso, precisamos da definição de termos semelhantes.
\begin{df}
Monômios são divididos em duas partes: a \emph{parte literal} e a \emph{parte numérica} ou \emph{coeficiente}.
\end{df}
\begin{df}
Dois termos são \emph{semelhantes} se, e somente se, a parte literal for igual.
\end{df}
Assim, as operações de adição e multiplicação são definidas a seguir.
\subsection{Adição de Polinômios}
Dados os polinômios $p(x)=a_0+a_1x+a_2x^2+\cdots+a_nx^n$ e $q(x)=b_0+b_1x+b_2x^2+\cdots+b_mx^m$, a soma é realizada entre os termos semelhantes. Assim, $p(x)+q(x)=c_0+c_1x+c_2x^2+\cdots+c_qx^q$, onde $c_i=a_i+b_i$.
\subsubsection{Propriedades da Adição}
A adição nos polinômios com coeficientes reais adquire as seguintes propriedades, herdadas a partir dos números reais:
\begin{itemize}
\item Comutativa
\item Associativa
\item Elemento Neutro $e(x)=0$
\item Elemento Oposto $-p(x)$
\end{itemize}

\subsection{Produto de Polinômios}
Dados os polinômios $p(x)=a_0+a_1x+a_2x^2+\cdots+a_nx^n$ e $q(x)=b_0+b_1x+b_2x^2+\cdots+b_mx^m$, o polinômio resultante é dado por $r(x)=c_0+c_1x^1+c_2x^2+\cdots+c_{m+n}x^{m+n}$, em que $c_i=(a_0b_i)+(a_1b_{i-1})+\cdots+(a_ib_0)$.
\begin{exemplo}
Dados $p(x)=x^3+2x-1$ e $q(x)=5x^2+10$, o produto $p(x)\cdot q(x)$ é dado por
\begin{align*}
p(x)\cdot q(x) &=(x^3+2x-1)\cdot q(x) \\
&=x^3(q(x))+2x(q(x))-1(q(x)) \\
&=x^3(5x^2+10)+2x(5x^2+10)-1(5x^2+10) \\
&=(5x^5+10x^3)+(10x^3+20x)+(-5x^2-10) \\
&=5x^5+10x^3+10x^3-5x^2+20x-10 \\
&=5x^5 + 20x^3 -5x^2 +20x -10
\end{align*}
\end{exemplo}
\subsubsection{Propriedades da Multiplicação}
O produto de polinômios com coeficientes reais possui as seguintes propriedades, herdadas dos números reais:
\begin{itemize}
\item Comutativa
\item Associativa
\item Elemento Neutro $p(x)=1$
\item Não possui divisores de zero
\end{itemize}

\subsection{Divisão Euclidiana em Polinômios}
Dados dois polinômios $p(x), d(x) \in \mathbb{R}[x]$, sendo $d(x) \neq 0$, existem dois polinômios $q(x), r(x) \in \mathbb{R}[x]$ tais que $p(x)=d(x)q(x)+r(x)$, com $r(x)=0$ ou  $\partial(r(x)) < \partial(d(x))$.
\begin{exemplo} A divisão de $p(x)=6x^3-3x^2-10x-12$ por $d(x)=x-2$ é dada por:
\[\polylongdiv[style=B]{6x^3-3x^2-10x-12}{x-2}\]
A divisão de $p(x)=x^5+2x^3-2x^2-3x+2$ por $d(x)=x^2-1$ é:
\[\polylongdiv[style=B]{x^5+2x^3-2x^2-3x+2}{x^2-1}\]
\end{exemplo}